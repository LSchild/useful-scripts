% !TeX program = lualatex

\documentclass[lualatex, 9pt]{beamer}

\usepackage{default}
\usepackage{geometry}
\usepackage{array}
\usepackage[notransparent, svgpath=./images/]{svg}
\usepackage{tikz}
\usepackage{comment}
\usepackage[utf8]{inputenc}
\usepackage{soul}
\usepackage{fontspec}
\usepackage{xcolor}
\usepackage{tabularx}
\usepackage{tkz-euclide}
\usepackage{caption}
\captionsetup[figure]{labelformat=empty}

\usepackage{transparent}

\setsansfont{Inconsolata Regular}

\usetikzlibrary{shadows,positioning,calc,decorations.pathreplacing}

\usetheme{CambridgeUS}
\usecolortheme{beaver}

%\usepackage[euler-digits]{eulervm}
\usefonttheme{professionalfonts}

\setbeamertemplate{itemize items}[triangle]

\newcommand\blfootnote[1]{%
	\begingroup
	\renewcommand\thefootnote{}\footnote{#1}%
	\addtocounter{footnote}{-1}%
	\endgroup
}

\definecolor{myred}{HTML}{8f0000}
\definecolor{lachs}{HTML}{d56d56}
\definecolor{blue1}{RGB}{72,145,211}
\definecolor{blue2}{RGB}{87,183,235}

\AtBeginSection[]{
	\begin{frame}
	\vfill
	\centering
	\setbeamercolor{coloredboxstuff}{fg=myred,bg=gray!20!white}
	\begin{beamercolorbox}[sep=8pt,center,shadow=false,rounded=true]{coloredboxstuff}
		\usebeamerfont{title}\insertsectionhead\par%
	\end{beamercolorbox}
	\vfill
\end{frame}
}

\begin{document}
	
	\tikzset{
		>=stealth',
		invisible/.style={opacity=0},
		visible on/.style={alt={#1{}{invisible}}},
		alt/.code args={<#1>#2#3}{
			\alt<#1>{\pgfkeysalso{#2}}{\pgfkeysalso{#3}}
		},	
}



\author[Leonard Schild]{Leonard Schild \\[0.8cm] Supervisors: Prof. Michael Backes and Kamil Kluczniak}
\title{Make FHE fast(er) \st{again} !}
\date{19.02.2020}
\institute{Saarland University}

\maketitle

\section{Motivation}

\begin{frame}
	\frametitle{Setting}
	\begin{center}
	\begin{tikzpicture}
		\centering
		\node[inner sep=0pt] (A) at (0,0.5) {\includesvg[width=0.08\textwidth]{avatar}};
		\node[inner sep=0pt] (B) at (-3,-2) {\includesvg[width=0.08\textwidth]{dna}};
		
		\node[inner sep=0pt] (b) at (-1.5, -2) {\includesvg[width=0.13\textwidth]{arrow1}};
		\node[inner sep=0pt, rotate=15] (c) at (1.75, -1.5) {\includesvg[width=0.13\textwidth]{arrow1}};
		\node[inner sep=0pt, rotate=-15] (d) at (1.75, -2.5) {\includesvg[width=0.13\textwidth]{arrow1}};
		
		\node[inner sep=0pt] (C) at (0,-2) {\includesvg[width=0.08\textwidth]{deep-learning}};
		\node[inner sep=0pt] (D) at (3.5,-1) {\includesvg[width=0.1\textwidth]{virus}};
		\node[inner sep=0pt] (E) at (3.5,-3) {\includesvg[width=0.1\textwidth]{bacteria}};
		
		%\draw[->, line width=0.5mm] (B) edge (C);
		%\draw[->, line width=0.5mm] (C) edge (D);
		%\draw[->, line width=0.5mm] (C) edge (E);
		
		\fill [fill=cyan, rounded corners, opacity=0] (-0.6,1.25) rectangle (0.6,-2.8);
		\fill [fill=cyan, rounded corners, opacity=0] (-3.6,1.25) rectangle (-2.4,-2.8);
		
	\end{tikzpicture}
	\end{center}
\end{frame}

\begin{frame}
\frametitle{Application}
\begin{center}
	\begin{overprint}
	\begin{tikzpicture}
	\centering
	\node[inner sep=0pt] (A) at (0,-2) {\includesvg[width=0.08\textwidth]{avatar}};
	
	\node[inner sep=0pt] (B) at (-3,-2) {{ \onslide<1>{\includesvg[width=0.08\textwidth]{hospital-building}} }};
	\node[inner sep=1.5pt, fill=white, fill opacity=0.7,text opacity=1, rounded corners, draw=blue1] (C) at (0.5,-1.5) {\includesvg[width=0.03\textwidth]{deep-learning}};
	\node[inner sep=0pt] (D) at (3.5,-1) {\includesvg[width=0.1\textwidth]{virus}};
	\node[inner sep=0pt] (E) at (3.5,-3) {\includesvg[width=0.1\textwidth]{bacteria}};
	
	\node[inner sep=0pt] (A) at (-1.6, -1.6) {{ \onslide<1>{\includesvg[width=0.05\textwidth]{dna}} }};
	
	\node[inner sep=0pt] (b) at (-1.5, -2) {\includesvg[width=0.13\textwidth]{arrow1}};
	\node[inner sep=0pt, rotate=15] (c) at (1.75, -1.5) {\includesvg[width=0.13\textwidth]{arrow1}};
	\node[inner sep=0pt, rotate=-15] (d) at (1.75, -2.5) {\includesvg[width=0.13\textwidth]{arrow1}};

	% transition 1
	\node[inner sep=0pt] (B) at (-2.92,-2) {{\onslide<2>{\includesvg[width=0.08\textwidth]{hospital-building-sad}} }};
	\node[inner sep=0pt] (A) at (-1.6, -1.6) {{ \onslide<2>{\includesvg[width=0.05\textwidth]{dna-sad}} }};
	\draw [decorate,decoration={brace,amplitude=10pt,mirror,raise=4pt},yshift=0pt, opacity=0]
	(-3.75,-2.75) -- (-0.6,-2.75);
	\draw [decorate,decoration={brace,amplitude=10pt,mirror,raise=4pt},yshift=0pt, visible on=<2>]
	(-3.75,-2.75) -- (-0.6,-2.75);
	
	%\node[inner sep=0pt, opacity=0] (F) at (-2.17, -4) {\includesvg[width=0.08\textwidth]{law}} ;
	\node[inner sep=0pt] (F) at (-2.17, -4) {{ \onslide<2>{\includesvg[width=0.08\textwidth]{law}} }} ;
	\fill [rounded corners, fill=red, opacity=0.2, visible on=<2>] (-3.75,-1.25) rectangle (-0.6,-2.75);
	
	\end{tikzpicture}
\end{overprint}
\end{center}
\end{frame}

\begin{frame}
\frametitle{Application}
\begin{center}
	\begin{overprint}
		\begin{tikzpicture}
		\centering
		\node[inner sep=0pt, visible on=<1>] (A) at (0,-2) {\includesvg[width=0.08\textwidth]{hospital-building}};
		\node[inner sep=0pt, visible on=<2>] (A) at (0,-2) {\includesvg[width=0.08\textwidth]{hospital-building-sad}};
		
		\node[inner sep=0pt] (B) at (-3,-2) {\includesvg[width=0.08\textwidth]{avatar} };
		\node[inner sep=1.5pt, fill=white, fill opacity=0.7,text opacity=1, rounded corners, draw=blue1] (C) at (0.5,-1.5) {\includesvg[width=0.03\textwidth]{dna}};
		\node[inner sep=0pt] (D) at (3.5,-1) {\includesvg[width=0.1\textwidth]{virus}};
		\node[inner sep=0pt] (E) at (3.5,-3) {\includesvg[width=0.1\textwidth]{bacteria}};
		
		\node[inner sep=0pt] (A) at (-1.6, -1.6) {{ \includesvg[width=0.05\textwidth]{deep-learning} }};
		
		\node[inner sep=0pt] (b) at (-1.5, -2) {\includesvg[width=0.13\textwidth]{arrow1}};
		\node[inner sep=0pt, rotate=15] (c) at (1.75, -1.5) {\includesvg[width=0.13\textwidth]{arrow1}};
		\node[inner sep=0pt, rotate=-15] (d) at (1.75, -2.5) {\includesvg[width=0.13\textwidth]{arrow1}};
		
		% transition 1
		%\node[inner sep=0pt] (B) at (-2.92,-2) {{\onslide<2>{\includesvg[width=0.08\textwidth]{hospital-building-sad}} }};
		%\node[inner sep=0pt] (A) at (-1.6, -1.6) {{ \onslide<2>{\includesvg[width=0.05\textwidth]{dna-sad}} }};
		\draw [decorate,decoration={brace,amplitude=10pt,mirror,raise=4pt},yshift=0pt, opacity=0]
		(-3.75,-2.75) -- (-0.6,-2.75);
		%\draw [decorate,decoration={brace,amplitude=10pt,mirror,raise=4pt},yshift=0pt, visible on=<2>]
		%(-3.75,-2.75) -- (-0.6,-2.75);
		
		%\node[inner sep=0pt, opacity=0] (F) at (-2.17, -4) {\includesvg[width=0.08\textwidth]{law}} ;
		%\node[inner sep=0pt] (F) at (-2.17, -4) {{ \onslide<2>{\includesvg[width=0.08\textwidth]{law}} }} ;
		
		\fill [rounded corners, fill=red, opacity=0.2, visible on=<2>] (-2.35,-1.15) rectangle (0.9,-2.75);
		
		\draw [decorate,decoration={brace,amplitude=10pt,mirror,raise=4pt},yshift=0pt, visible on=<2>]
		(-2.35,-2.75) -- (0.9,-2.75);
		
		\node[inner sep=0pt] (F) at (-0.725, -4) {{ \onslide<2>{\includesvg[width=0.08\textwidth]{leak} \includesvg[width=0.08\textwidth]{server}}  }};
		
		\end{tikzpicture}
	\end{overprint}
\end{center}
\end{frame}

\begin{frame}
\frametitle{No solution ?}
\begin{center}
	\begin{tikzpicture}
	\node [inner sep=10pt, fill=black, fill opacity=0.3, text opacity=0.95, rounded corners] {\includesvg[width=0.3\textwidth]{avatar}};
	\end{tikzpicture}
\end{center}
\end{frame}

\begin{frame}
	\frametitle{}
	\begin{center}
		\Huge{Fully Homomorphic encryption}
	\end{center} 	
\end{frame}

\begin{frame}
\frametitle{Fully Homomorphic Encryption}
\begin{overprint}

\begin{definition}[Homomorphism]
Given two groups $\mathbb{G} = (G,\circ)$ $\mathbb{H} = (H,\bullet)$, $f:G \rightarrow H$ is a group homomorphism if:
$$\forall a,b \in G:  f(a \circ b) = f(a) \bullet f(b)$$
\end{definition}
\vspace{.75cm}
\onslide<2->{
Fully Homomorphic Encryption:}
\begin{align*}
\onslide<2->{&\left\{ \begin{array}{cc} Enc(a + b) = Enc(a) \oplus Enc(b) \\
Enc(a \cdot b) = Enc(a) \odot Enc(b) \end{array} \right\}} \\[0.5cm]
\onslide<3->{\Leftrightarrow &\left\{ \begin{array}{cc} a + b = Dec(Enc(a) \oplus Enc(b)) \\
a \cdot b = Dec(Enc(a) \odot Enc(b)) \end{array}
\right\}}
\end{align*}

\end{overprint}

\end{frame}

\begin{frame}
\frametitle{Setting: Problem Solved !}
\begin{center}
	\begin{tikzpicture}
	\centering
	\node[inner sep=0pt] (A) at (0,-2) {\includesvg[width=0.08\textwidth]{avatar}};
	
	\node[inner sep=0pt] (B) at (-3,-2) {\includesvg[width=0.08\textwidth]{hospital-building}};
	\node[inner sep=1.5pt, fill=white, fill opacity=0.7,text opacity=1, rounded corners, draw=blue1] (C) at (0.5,-1.5) {\includesvg[width=0.03\textwidth]{deep-learning}};
	\node[inner sep=0pt] (D) at (3.5,-1) {\includesvg[width=0.1\textwidth]{virus}};
	\node[inner sep=0pt] (E) at (3.5,-3) {\includesvg[width=0.1\textwidth]{bacteria}};
	
	\node[inner sep=0pt] (A) at (-1.6, -1.5) {\includesvg[width=0.05\textwidth]{dna}};
	\node[inner sep=0pt] (Ab) [left = 0.01em of A] {$\left[\vphantom{\includesvg[width=0.03\textwidth]{dna}}\right.$};
	\node[inner sep=0pt] (Aa) [right = 0.01em of A] {$\left.\vphantom{\includesvg[width=0.03\textwidth]{dna}}\right]_{p_H}$};
	%\node[inner sep=0pt] (Ab) at (-1.6, -1.6) {\includesvg[width=0.05\textwidth]{dna}};
	
	\node[inner sep=0pt] (b) at (-1.5, -2) {\includesvg[width=0.13\textwidth]{arrow1}};
	\node[inner sep=0pt, rotate=15] (c) at (1.75, -1.5) {\includesvg[width=0.13\textwidth]{arrow1}};
	\node[inner sep=0pt, rotate=-15] (d) at (1.75, -2.5) {\includesvg[width=0.13\textwidth]{arrow1}};
	
	\node[inner sep=0pt] (Ab) at (2.7,-2) {$\left[\vphantom{\includesvg[width=0.17\textwidth]{dna}}\right.$};
	\node[inner sep=0pt] (Ab) at (4.5,-2) {$\left.\vphantom{\includesvg[width=0.17\textwidth]{dna}}\right]_{p_H}$};
	
	\end{tikzpicture}
\end{center}
\end{frame}

\begin{frame}
	\frametitle{Issues}
\begin{center}
	Why hasn't the industry created a fancy new buzzword yet ? \\[1cm]
\end{center}
\begin{center}
	\begin{overprint}
\begin{tikzpicture}
	\node [inner sep=0pt] (A) at (-2.5,0) {{ \onslide<2-3>{\includesvg[width=0.1\textwidth]{memory}} }};
	\node [inner sep=0pt, opacity=0.2] (A) at (-2.5,0) {{ \onslide<4->{\includesvg[width=0.1\textwidth]{memory}} }};
	
	\node [inner sep=0pt] (B) at (2.5,0) {{ \onslide<3->{\includesvg[width=0.1\textwidth]{speed}} }};
	\node [inner sep=0pt, align=center] (C) at (-2.5,-1.5) {{\onslide<2-3>{Multiple keys of}} \\ {\onslide<2-3>{large size} }};
	\node [inner sep=0pt, align=center, opacity=0.2] (C) at (-2.5,-1.5) {{\onslide<4->{Multiple keys of}} \\ {\onslide<4->{large size} }};
	%\node [inner sep=0pt, align=center, opacity=0.2] (C) at (-2.5,-1.5) {Multiple keys of \\ large size};
	\node [inner sep=0pt, align=center] (D) at (2.5,-1.5) {{ \onslide<3->{Slow operations} }};
	\draw [very thick, draw=green, opacity=0.5, rounded corners, text opacity=1, visible on=<4->] (1,1) rectangle (4,-2);
\end{tikzpicture}\\[0.75cm]
% mention MPC vs FHE tradeoffs
\onslide<5->{Alternative solution: Multi Party Computation}
	\end{overprint}
\end{center}
\end{frame}

\section{Thesis}

\begin{frame}
\frametitle{Lattices}

\begin{overprint}
	\onslide<5>{\blfootnote{https://web.eecs.umich.edu/\texttt{\sim}cpeikert/pubs/lattice-survey.pdf}}
\begin{Definition}
	A Lattice $\mathcal{L} \subset \mathbb{R}^n$ is:
	\begin{itemize}
		\item<2-> an \textit{additive subgroup}: $\mathbf{0} \in \mathcal{L}, -x, x+y \in \mathcal{L}$ for all $x,y \in \mathcal{L}$.
		\item<3-> \textit{discrete}: every $x \in \mathcal{L}$ has a neighborhood in $\mathbb{R}^n$ in which $x$ is the only lattice point.
	\end{itemize}
\end{Definition}
\onslide<4->{
Every lattice can be finitely  generated  as  the  integer  linear  combinations  of  some  linearly  independent basis vectors $\mathbf{B} = \{ \mathbf{b}_1,...,\mathbf{b}_k \}$. I.e.}
\onslide<5->{
\begin{align*}
	\mathcal{L} = \left\{\sum_{i=1}^{k} z_i \cdot \mathbf{b}_i : z_i \in \mathbb{Z}  \right\}
\end{align*}}
\end{overprint}
\end{frame}

\begin{frame}
\frametitle{Lattices}
\begin{overprint}
\begin{align*}
\mathbf{B} = \left\{ \left(\begin{array}{c}
3\\1\end{array}\right), \left(\begin{array}{c}
1\\2
\end{array}\right) \right\} 
\end{align*}
\onslide<2->{
\begin{figure}[h]
	\includesvg[width=0.7\textwidth]{lattice}
\end{figure}
}
\end{overprint}

\end{frame}



\begin{frame}
\frametitle{Lattices: Closest Vector Problem}
\begin{align*}
\mathbf{B} = \left\{ \left(\begin{array}{c}
3\\1\end{array}\right), \left(\begin{array}{c}
1\\2
\end{array}\right) \right\} 
\end{align*}\vspace{0em} %% Necessary, but WTF ???
\begin{figure}[h]
	\includesvg[width=0.7\textwidth]{latticecvp}
\end{figure}
\end{frame}

\begin{frame}
	\frametitle{Lattices: Learning With Errors}
	\begin{overprint}
		Introduced in \cite{reg05}. \\[1cm]
		\only<1>{
			\begin{align*}
			14s_1 + 15s_2 + 5s_3 + 2s_4 &= 8 \pmod{17} \\
			13s_1 + 14s_2 + 14s_3 + 6s_4 &= 16 \pmod{17} \\
			6s_1 + 10s_2 + 13s_3 + 1s_4 &= 3 \pmod{17} \\
			10s_1 + 4s_2 + 12s_3 + 16s_4 &= 12 \pmod{17} \\
			9s_1 + 5s_2 + 1s_3 + 6s_4 &= 9 \pmod{17} \\
			\vdots
			\end{align*}
		}
		\only<2->{
		\begin{align*}
		14s_1 + 15s_2 + 5s_3 + 2s_4 &= 8 + e_1\pmod{17} \\
		13s_1 + 14s_2 + 14s_3 + 6s_4 &= 16 + e_2\pmod{17} \\
		6s_1 + 10s_2 + 13s_3 + 1s_4 &= 3 + e_3\pmod{17} \\
		10s_1 + 4s_2 + 12s_3 + 16s_4 &= 12 + e_4\pmod{17} \\
		9s_1 + 5s_2 + 1s_3 + 6s_4 &= 9 + e_5\pmod{17} \\
		\vdots
		\end{align*}
		}
	\end{overprint}
\end{frame}

\begin{frame}
	\frametitle{Lattices: FHE}
	\begin{overprint}
	A bulk of FHE schemes are based on lattice problems or problems closely related to them (e.g LWE).\\[1cm]

	\onslide<2->{Most rely on easily distributable computations, e.g.:}
	\begin{itemize}
		\item<3-> Matrix products
		\item<4-> Vector addition or multiplication
		\item<5-> Polynomial multiplication (R-LWE)
	\end{itemize} 
	\vspace{3em}
	\onslide<6-> $\Rightarrow$ Use GPU to parallelise computation !
	\end{overprint}
\end{frame}



\begin{frame}
	\frametitle{GPGPU programming}
	False equivalency: CPU and GPU core
	\vfill
	\begin{center}
	\begin{minipage}{0.4\textwidth}
		\centering
		\includesvg[width=0.3\textwidth]{electronic} \\[0.5cm]
		\begin{itemize}
		\item<2-> Low latency \\
		\item<3-> Few cores\\
		\item<4-> High clock frequency
		\end{itemize}
	\end{minipage}\begin{minipage}{0.4\textwidth}
	\centering
	\includesvg[width=0.4\textwidth]{gpu} \\[0.5cm]
	\begin{itemize}
	\item<2-> High throughput \\
	\item<3-> Large amount of cores\\
	\item<4-> Low clock frequency
	\end{itemize}
\end{minipage}
\end{center}

\end{frame}
\begin{comment}

\begin{frame}
	\frametitle{A more concrete view}
	Goal: Make FHE practical for real life applications.\\
	\begin{itemize}
		\item<2-> \cite{chillotti2016faster,chillotti2019tfhe} TFHE: State of the art FHE scheme \begin{enumerate}
			\item<3-> Based on \cite{gentry2013homomorphic}
			\item<3-> Intuition: $Enc(\mu,\vec{s}) = (\vec{a},\langle \vec{a},\vec{s} \rangle + \mu + e)$
			\item<3-> Assumption: Torus-LWE\\[0.5cm]
		\end{enumerate} 
		\item<4-> \cite{bourse2018fast} FHE on NN \begin{enumerate}
			\item<5-> Uses TFHE
			\item<5-> Discrete NN to recognize digits
			\item<5-> Runs on encrypted data\\[0.5cm]
		\end{enumerate}
		\item<6-> NVIDIA: CUDA \begin{enumerate}
			\item<7-> Cluster uses Tesla K80
			\item<7-> Optimized libraries for NVIDIA GPUs.
			\item<7-> Outdated OpenCL version (1.2)\\[0.1em]
			\item<7->[] {\tikz\node[inner sep=0pt, opacity=0.5]{Can use PTX assembly};}
		\end{enumerate}
	
	\end{itemize}
\end{frame}

\end{comment}

\begin{frame}
\frametitle{A more concrete view: Scheme}
	\begin{figure}[h]
		\centering
	\includegraphics[width=0.3\linewidth]{images/logo-tfhe.png}
	\caption{\cite{chillotti2019tfhe,chillotti2016faster}}
	\end{figure}
	\begin{itemize}
		\item<2-> Based on \cite{gentry2013homomorphic}
		\item<2-> Intuition: $Enc(\mu,\vec{s}) = (\vec{a},\langle \vec{a},\vec{s} \rangle + \mu + e)$
		\item<2-> Assumption: Torus-LWE\\[0.5cm] 
	\end{itemize}
\end{frame}

\begin{frame}
\frametitle{A more concrete view: Network}
\begin{figure}[h]
	\centering
	\includesvg[width=0.2\linewidth]{images/deep-learning.svg}
	\caption{\cite{bourse2018fast}}
\end{figure}
\begin{itemize}	
	\item<2-> Recognize handwritten digits
	\item<2-> Discrete Neural Network
	\item<2-> Uses TFHE\\[0.5cm]
\end{itemize}
\end{frame}

\begin{frame}
\frametitle{A more concrete view: GPU Programming}
\begin{figure}[h]
	\centering
	\includegraphics[width=0.4\linewidth]{images/cuda.jpg}
\end{figure}
\begin{itemize}	
	\item<2-> CISPA Cluster uses Tesla K80
	\item<2-> Optimized libraries for NVIDIA GPUs.
	\item<2-> Outdated OpenCL version (1.2)\\[0.1em]
	\item<2->[] {\tikz\node[inner sep=0pt, opacity=0.5]{Can use PTX assembly};}
\end{itemize}
\end{frame}

\begin{frame}
	\frametitle{Timeline}
	Two overlapping parts: \\[1cm]
	\begin{itemize}
		\item<2-> Implementation \begin{enumerate}
			\item Port TFHE to GPU \\
			\item Implement basic optimizations (e.g polynomial multiplication) \\
			\item Modify existing algorithms to make them distributable. \\[0.5cm]
		\end{enumerate}
		\item<3-> Evaluation \begin{enumerate}
			\item Repeat \cite{bourse2018fast} experiment and measure speedup 
			\item Implement other neural network which could be used in real life and test our methods. 
		\end{enumerate}
	\end{itemize}
\end{frame}




\begin{frame}[allowframebreaks]
\frametitle{References}
\bibliographystyle{amsalpha}
\bibliography{refs.bib}
\end{frame}

\end{document}
